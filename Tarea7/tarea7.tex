\documentclass[12pt,a4paper]{report}

\usepackage[left=2cm, right=2cm, top=4cm, bottom=2cm]{geometry}
\usepackage{enumitem}
\usepackage{fontspec}
\usepackage{tikz}
\usepackage{amsmath}
\usepackage{algpseudocode}

%Eventos
\algblockdefx[Initially]{Initially}{EndInitially}{\textbf{initially do}}{\textbf{end initially}}
\algblockdefx[Upon]{Upon}{EndUpon}[1]{\textbf{upon #1}}{\textbf{end upon}}

\begin{document}
	%Portada
\begin{titlepage}
	\centering
	{\scshape\LARGE Universidad Nacional Autónoma de México \par}
	\vspace{1cm}
	{\scshape\Large Computación Distribuida\par}
	\vspace{1.5cm}
	{\huge\bfseries Tarea 7\par}
	\vspace{.5cm}
	{\Large\itshape Edgar Quiroz Castañeda \par}
	\vspace{.5cm}
	{\Large\itshape Jerónimo Almeida Rodríguez \par}
	\vfill
	 \includegraphics[width=0.5\textwidth]{escudo_f-ciencias.png}
	\vfill

	{\large Jueves 1 de noviembre del 2018 \par}
\end{titlepage}

\pagebreak
\setlength{\voffset}{-0.75in}
\setlength{\headsep}{5pt}

%Ejercicios
\begin{enumerate}
	%1
	\item {
		Considera la ejecución $\alpha$ de la Figura 1. Los retardos máximos
		para cada uno de los eventos son $B_{i, j}$, donde $i$ es el evento
		que envía y $j$ el que recibe.
		\begin{enumerate}
			%a
			\item {
				¿Cuánto vale el retardo máximo de $c$ a $e$ en términos de
				las $B$s? Es decir, dado $real(c)-real(e) \leq x$, ¿cuánto
				vale $x$?\\
				$B_{cd} + B_{de}$
			}

			%b
			\item {
				Anotar los eventos con los relojes escalares y los relojes
				vectoriales.\\
				$a:1, b:2, c:3, d:4, e:5$\\
				$a:(1, 0, 0), b:(1, 1, 0), c:(1, 1, 1), d:(1, 2, 1), e:(2, 2,
                1)$
			}

			%c
			\item {
				¿Cuáles son eventos concurrentes y cuáles no?\\
				El evento $a\rightarrow e $ es concurrente con cualquier otro
                evento, pues no hay una trayectoria que una esta arista con
                cualquier otra en la gráfica de causalidad. Por otro lado,
                ningún evento que suceda en la trayectoria $T=\{a,b,c,d,e\}$ es
                concurrente co algún otro evento, pues hay una trayectoria
                dirigida de algún evento en la trayectoria a cualquier otro que
                suceda después.
			}

			%d
			\item {
				Describe todos los cortes consistentes de $\alpha$.\\
				Si en evento está en $S$, entonces ninguno de los eventos
				que causa puede estar en $T$. Entonces, para generar los
				eventos en $S$, se toma un evento y todos los que este causa,
				poniendo a todos los demás en $T$.\\
				\begin{tabular}{| l | c |}
					\hline
						S & T \\ \hline
						e & a, b, c, d \\ \hline
						d, e & a, b, c \\ \hline
					    c, d, e & a, b \\ \hline
					    a, b, c, d, e & a \\
					\hline
				\end{tabular}
			}
		\end{enumerate}
	}

	%2
	\item {
		Sea $G = (V, E)$ la gráfica dirigida de causalidad de una ejecución
		$\alpha$, $e, e' \in V$ y $L$ un reloj vectorial.\\
		Un evento causó a otro si existe un camino dirigido que inicia en
		el primer evento y acaba en el segundo.\\
		Demuestra que $e$ causó a $e'$ si y sólo si $L(e) < L(e')$.
    \begin{itemize}
        \item[\rightarrow]{
            Sabemos que si $e$ causó a $e'$ entonces $e$ tuvo que haber
            modificado su reloj vectorial aumentadolo en $1$ antes de enviarlo.
            Entonces, el reloj vectorial para el evento $e$ antes de que
            sucediera el siguiente evento es menor que el del evento que causó,
            pues tuvo que aumentar el contador de su evento antes de que se
            ejecutara. Así, si un evento $e$ causó a un evento $e'$, entonces,
            particularmente el reloj vectorial que el evento $e$ envió a $e'$
            es menor que el que tiene $e'$, aún cuándo $e$ no haya causado
            directamente a $e'$ (es decir que si $e$ es el evento $n$, entonces
            $e'$ sería el evento $n+1$), pues según cómo están definidos estos
            relojes, el reloj solamente puede ir aumentando los valores de cada
            entrada.\\
            Por lo tanto, si $e$ causó a $e'$, entonces $L(e) < L(e')$.
        }
        \item[\leftarrow]{
            Que $L(e) < L(e')$ significa que cada entrada del vector $L(e)$ es
            menor o igual que cada entrada del vector $L(e')$. Esto implica que,
            cómo ningún proceso puede aumentar el contador de otro proceso,
            entonces para que $L(e) < L(e')$, el proceso de $e$ tuvo que haber
            aumentado su contador en algún momento antes que $e'$ fuera
            ejecutado.\\
            Basados en esto, sabemos que $e$ causó a $e'$ porque de no ser así
            no habría manera de que el reloj de $e'$ sea mayor que el reloj de
            $e$, pues ningún evento anterior a $e'$ hubiera aumentado el
            valor del proceso en el que ocurrió $e$. Entonces, al $e$ aumentar
            el valor del proceso en el que ocurrió en el reloj vectorial lo pasa
            al siguiente evento. Si este evento es $e'$, la propiedad se cumple.
            De otra manera, por inducción notamos que cómo el reloj del evento
            anterior a $e'$ es menor que el de $e'$ el reloj del evento anterior
            a este es menor y eventualmente tendríamos que llegar al evento
            $e$, pues si no, no habría manera de que $e'$ hubiera podido recibir
            el reloj aumentado que $e$ envió.
            Por lo tanto, si $L(e) < L(e')$, entonces $e$ causó a $e'$.
        }
    \end{itemize}
	}

	%3
	\item {
		Lee los primeros 5 sueños del libro \textit{Einstein's Dream}.
		Redacta un pequeño resumen de cada uno de ellos.
        \begin{itemize}
            \item{\textbf{14 April 1905}\\
                En este primer sueño, se relata una linea de tiempo circular,
                es decir que todo lo que secede, sucedió y está por suceder ya
                lo hizo una cantidad desconocida de veces. Según el cuento, la
                gente, en general no es consciente de esto y vive su vida
                atesorando cada momento cómo si nunca fuera a volver, celebrando
                la unicidad de una primera vez cómo si nunca mas volviera a
                suceder, sufriendo la pérdida de un ser querido cómo si nunca
                más lo fueran a volver a ver.\\
                Pero, de vez en cuando, algunos escogidos recuerdan en
                tormentosos sueños aquellos errores que cometieron y
                malfortunios que vivieron ya tantas veces y que no van a poder
                hacer nada para cambiarlos.\\
            }
            \item{\textbf{16 April 1905}\\
                En este sueño, el tiempo es cómo un riachuelo, fluye en una
                dirección pero a veces arrastra algo inesperado.\\
                Tal es el caso de una viajera anónima del futuro que quedó
                atrapada en este flujo. El problema es que esta viajera sabe
                que levantar una partícula de polvo puede tener efectos tan
                catastróficos cómo la creación de la Union Europea en las
                décadas por venir o la falta de ella. Atormentada por esto,
                encuentra un rincón oscuro dónde no pueda hacer nada que afecte
                el futuro de dónde ha venido.\\
                La gente pasa y, cómo a muchos desafortunados cómo ella, los ven
                y se conmiseran de ellos, sabiendo que no hay manera de
                ayudarlos, pues al hacerlo pueden afectar un futuro que ellos
                mismos no conocen.\\
            }
            \item{\textbf{19 April 1905}\\
                Este escrito nos relata la historia de un hombre y los mundos
                que se bifurcan a partir de una decisión que toma.\\
                En el primer mundo, decide asentarse, trabajar duro, disfrutar
                momentos con sus amigos. Un día conoce a una mujer, la corteja
                a lo largo de unos meses y envejecen juntos.\\
                En el segundo mundo, decide hablar con la mujer de sus sueños.
                Su sonrisa, su risa, su elocuencia al hablar lo enamoran. Lo
                convence de cambiar de ciudad para vivir con ella y viven una
                vida apacionada. Él es feliz.\\
                En el tercer mundo, igual que en el segundo, decide hablar con
                ella, pero ella lo rechaza.\\
                Así se desarrolla la vida en este universo. Cada decisión que se
                toma se bifurca en otros mundos en los que la historia se
                desarrolla de manera diferente; y hay una infinidad de ellos.\\
            }
            \item{\textbf{24 April 1905}\\
                En el cuarto sueño vemos dos interpretación del tiempo. En la
                primera, el tiempo se mide de manera "orgánica", es decir, las
                personas que viven bajo esta métrica de tiempo, comen cuándo
                están hambrientas, trabajan cuándo se despiertan y, en general,
                viven el momento cuando llega. Los relojes existen, pero ellos
                solamente los usan cómo ornamentos; no se rigen por ellos.\\
                por otro lado, están aquellas personas que viven regidas por un
                tiempo "mecánico"; se lavantan puntualmente a las 7:00,
                desayunan y se van a trabajar, almuerzan a mediodía y cenan en
                la noche. Incluso tienen horarios para convivir. Ellos creen que
                los impulsos naturales, cómo el hambre solamente son impulsos
                químicos que deben ser sometidos a este horario mecánico.\\
                Se puede vivir en cualquiera de ellos, pero no en ambos. Los dos
                son verdaderos, pero son distintos.
            }
            \item{\textbf{26 April 1905}\\
                Aquí el autor nos cuenta un sueño que, de los cinco que había
                que leer para esta tarea, creo que es el que mejor podemos
                aplicar al mundo actual. En este sueño, se malinterpretó un
                resultado científico y la gente empezó a creer que entre más
                alto vivía, podría preservar su juventud por más tiempo. Así,
                la gente se empezó a mover de los valles hacia las montañas y
                el estatus social estaba determinado por que tan alto alguien
                vivía.\\
                A pesar de las complicaciones que conlleva vivir en las alturas
                cómo que el aire es muy delgado y los alimentos son más escasos
                y hay menos variedad. Por esta razón, la gente de las alturas
                se ve más vieja de lo que es en realidad.\\
                Hay a quienes no les importa esta regla social y prefieren vivir
                en el momento y disfrutar de lo que las planicies ofrecen:
                árboles que den sombra, lagos dónde nadar, etc. \\
                Con todo esto, la gente de este sueño vive regida por esta norma
                social de vivir lo más alto posible.
            }
        \end{itemize}
	}

	%4
	\item {
		Sean $A$ y $B$ dos procesos cuyos relojes no están sincronizados,
		pero ambos tienen un $drift$ acotado. Un mensaje de $A$ a $B$
		tarda a lo más $D$, en tiempo real.\\
		Propón un algoritmo para que $B$ estime el tiempo de llegada de un
		mensaje de $A$, si este manda mensajes cada $T$ unidades de tiempo.

		\begin{algorithmic}[1]
			\Initially
			 	\State maxDelay = 0
				\State lastSeen = 0
			\EndInitially
			\Statex

			\Upon{receiving $m$ from $A$}
				\If{getTime() - lastSeen > maxDelay}
				    \State maxDelay = getTime() - lastSeen
				\EndIf
				\State lastSeen = getTime()
				\State print('next message at', getTime() + maxDelay)
			\EndUpon
		\end{algorithmic}
	}
\end{enumerate}

\end{document}
