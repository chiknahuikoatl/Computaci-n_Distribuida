\documentclass[12pt,a4paper]{report}

\usepackage[utf8x]{inputenc}
\usepackage[left=2cm, right=2cm, top=5cm]{geometry}
\usepackage{enumitem}
\usepackage{fontspec}
\usepackage{tikz}
\usepackage{amsmath}
\usepackage{algpseudocode}

\algblock[Name]{Start}{End}
\algblockdefx[NAME]{START}{END}
[2][Unknown]{Start #1(#2)}
{Ending}
\algblockdefx[NAME]{}{OTHEREND}
[1]{Until (#1)}

\usetikzlibrary{graphs, shapes, snakes, graphdrawing}
\usegdlibrary{layered, force}

\begin{document}

	\newcommand{\upon}[1]{\textbf{Upon} #1 \textbf{do}}

	\begin{titlepage}
		\centering
		{\scshape\LARGE Universidad Nacional Autónoma de México \par}
		\vspace{1cm}
		{\scshape\Large Computación Distribuida\par}
		\vspace{1.5cm}
		{\huge\bfseries Tarea 5\par}
		\vspace{.5cm}
		{\Large\itshape Edgar Quiroz Castañeda \par}
	    \vspace{.5cm}
		{\Large\itshape Jerónimo Almeida Rodríguez \par}
		\vfill
		 \includegraphics[width=0.5\textwidth]{escudo_f-ciencias.png}
		\vfill

		{\large Jueves 27 de septiembre del 2018 \par}
	\end{titlepage}

	\pagebreak
	\setlength{\voffset}{-0.75in}
	\setlength{\headsep}{5pt}

	\begin{enumerate}
		%Ejercicio 1
		\item {
			Considere la gráfica $G$. Propón retardos y ejecuta el algoritmo PIF con
			$p_1$ como raíz.\\
		}

		%Ejercicio 2
		\item {
			En la Ciudad de México varios puntos de control (nodos) fueron agregados.\\
			Cada uno de estos puntos tiene información de cuantas personas viven
			dentro de cierto diámetro. \\
			Un punto (raíz) de la delegacion Benito Juárez quiere recolectar la
			información de todos los puntos de control. \\
			Escribe un algoritmo para que cada nodo recolecte la información de sus
			hijos para enviarsela al padre sin repetirla.\\

			\begin{algorithmic}[1]
					\State $children \ \leftarrow \emptyset$
					\State $nonChildren \ \leftarrow \emptyset$
					\If{$p_{id}\ =\ root$}
						\State $parent\ \leftarrow\ root$
						\State $send$ $M$ to all neighbors
					\Else
						\State $parent\ \leftarrow\ \bot$
					\EndIf
					\State \upon{receiving $M$ from $p$}
					\Start
						\If{$parent\ =\ \bot$}
							\State $parent\ \leftarrow\ p$
							\State $send$ $M$ to all neighbors except $p$
						\Else
							\State $send$ $nack$ to $p$
						\EndIf
					\End
					\State \upon{receiving $nack$ from $p$}
					\Start
						\State $nonChildren\ \leftarrow\ nonChildren\ \cup\ {p}$
					\End
					\textbf{When} $children\ \cup\ nonChildren\ =\ allMyNeighbours
								\textbf{do}$
					\Start
						\If{$p_{id}\ =\ root$}
							\State END
						\Else
							\State $send\ ack\ to\ parent$
						\EndIf
					\upon{receiving $ack$ from $k$}
					\Start
						\State $add\ k\ to\ children$
					\End
					\End
			\end{algorithmic}
			}

		%Ejercio 3
		\item{
			Demuestre por inducción que en el algoritmo PI, cada nodo $v$ recibe el
			mensaje $M$ por primera vez en a lo más $d(root, v)$.\\

			Por inducción sobre el orden en el que los nodos reciben $M$.\\
			Caso base\\
			Sea $v \in V$ el primer nodo que recibe $M$. Digamos que lo recibe en $t(v)$.\\
			Como es el primero en obtenerla, tenemos que
			Supongamos por contradicción que $t(v) > d(root, v)$. Entonces, tenemos que


			Hipótesisi de inducción.\\
			Todos los nodos que reciben $M$ antes del lugar $k$ cumplen que
			$d(root, v) = t(v)$.\\

			Paso inductivo\\
			Sea $v \in V$ el vértice que recibe $M$ en el lugar $k$.

		}
		%Ejercio 4
		\item{
			Se dice que una digráfica $G = (V, E)$ tiene raiz $r \in V$ y cada
			vértice $v \in V$ es alcanzable desde $r$, es decir, existe un camino
			dirigido alcanzable desde $r$ (un camino dirigido que empieza en $r$ y
			termina en $v$).\\
			Una digráfica $G$ es un  ́arbol dirigido si tiene raíz y la versión no dirigida
			de $G$ es un árbol. Demuestra el siguiente teorema:\\

			\textbf{Teorema 1}\\
			Sea G una digráfica. Las siguientes condiciones son equivalentes
			\begin{enumerate} [label = \alph*)]
				\item {
					$G$ es un árbol dirigido.\\
				}
				\item {
					$G$ tiene una raíz desde la cual hay un único camino dirigido a
					vértice.\\
				}

				\item{
					$G$ tiene una raíz $r$ para la cual $\delta(r)_{in} = 0$, y para cualquier
					otro vértice $v$, $\delta_{in}(v) = 1$.\\
				}

				\item{
					$G$ tiene una raíz $r$ y el quitar cualquier arista interumpe la
					condición.\\
				}
				\item{
					La versión no dirigida de $G$ es conexa y $G$ tiene un vértice $r$ para
					el cual $\delta_{in}(r) = 0$ mientras que para cualquier otro vértice
					$v$ $\delta(v)_{in} = 1$.\\
				}
			\end{enumerate}

			$a \implies b$\\
			Sea $G$ un árbol dirigido. Por definición, significa que tiene una raíz,
			esto es un vértice desde el cuál hay un camino a cualquier otro vértice de
			$G$.\\
			Sólo falta ver que ese camino es único.\\

			Primero, notemos que esos camino son trayectorias, pues si uno de estos
			caminos, dígamos $c = (r = v_1, ..., v_n)$ tuviera un vértice repetido,
			digamos $v_i = v_j, i < j$, entonces $c' = (r = v_1, ..., v_i = v_j, ... v_n)$
			sería un cliclo en la versión no dirigida de $G$, por lo que la versión
			no dirigida de $G$ no sería un árbol.\\

			Luego, tomemos dos de estas trayectorias $c_1 = (r = v_1, ..., v_n = u)$ y
			$c_2 = (r = w_1, ..., w_m = u)$ y supongamos que son diferentes.\\
			Sea $v_i = w_j \in V$ el primer vértice no raíz que hay en común entre
			$c_1$ y $c_2$, cuya existencia se puede garantizar porque en el caso
			extremo este vértice es $u$.\\
			Entonces, en la versión no dirigida de $G$, consideremos el camino
			$c_3 = (r = v_1, v_2, ..., v_i = w_j, w_{j-1}, ..., w_2, w_1 = r)$.\\
			Cómo los $v_k$ provienen de una trayectoria, entonces no se repiten entre
			sí, al igual que los $w_l$.\\
			Además, como $v_i = w_j$ es el primer vértice en común entre las
			trayectorias originales, entonces los $v_k$ y  los $w_l$ no tienen ningún
			otro vértice en común.
			Entonces $c_3$ es una trayectoria, y comienza  y termina en el mismo
			vertice, por lo que es un ciclo.\\
			Entonces, la versión no dirigida de $G$ no es un árbol, lo cuál no es
			posible. Esta contradicción surge de suponer que $c_1$ y $c_2$ eran
			diferentes. Entonces, cualesquiera dos caminos entre la raiz y un mismo
			vértice son en realidad el mismo.\\
			Entoces el camino dirigido entre la raíz y cualquier vértice es único.\\

			$b \implies c$\\
			Sea $G$ una digráfica dirigida con una raíz $r$ y cuyos caminos entre la raíz y
			cualquier otro vértice son únicos.\\
			Veamos que $\delta(r)_{in} = 0$. Por contradicción, supongamos que
			$\delta(r)_{in} < 0$. Entonces sea $v \in N^+(r)$. Como $r$ es raíz,
			entonces existe un camino, $c = (r = v_1, ..., v)$. \\
			Entonces, $c' = (r = v_1, ..., v, r)$ contiene un ciclo en la versión no
			dirigida de $G$, pero esta debe de er un árbol.\\
			Entonces $\delta(r)_{in} = 0$.\\
			Luego, veamos cuál es el grado interior de los demás vértices.
			Sea $w \in V$. Como hay un camino $c_1 = (r = w_1, ..., w_n = w)$,
			entonces al menos un vértice $w_{n-1}$ incide en $w$, por lo que
			$\delta(w)_{in} \geq 1$.\\
			Luego, supongamos que $\delta(w)_{in} > 1$. Entonces sean
			 $u_1, u_2 \in N^+(w), u_1 \neq u_2$.\\
			Como $r$ es raíz, entonces existen caminos únicos $s_1 = (r = x_1, ..., x_n = u_1)$
			y $s_2 = (r = y_1, ..., y_n = u_2)$.\\
			Tanto $s_1$ como  $s_2$ inciden es $w$, entonces
			$s_1 = (r = x_1, ..., x_n = u_1, w)$ $s_2 = (r = y_1, ..., y_n = u_2, w)$
			son dos caminos de la raiz a $w$ diferentes, pues al menos $x_n = u_1 \neq y_n = u_2$.
			Pero los caminos desde la raíz son únicos. Esta contradicción surge de
			asumir que $\delta(w)_{in} > 1$.\\
			Por lo tanto, $\delta(w)_{in} = 1$.\\

			$c \implies d$\\
			Sea $G$ una digráfica que tiene una raíz $r$ para la cual $\delta(r)_{in} = 0$,
			y para cualquier otro vértice  $v \in V$, $\delta_{in}(v) = 1$.\\
			Cómo $G$ tiene raíz, sólo falta demostrar que $G-\{e\}, e \in E$ no tiene
			raíz.\\
			Sea $e = (v, w) \in E$. Si $v = r$, entonces $(v, w)$ era el camino único
			entre $v$ y $w$. Entonces ya no hay un camino entre $v$ y $w$. Entonces
			$r$ ya no es raíz.\\
			Si $r \neq w$, $\delta_{in}(w) = 1$, esto es que sólo un vértice
			incide en $w$, y cómo $v$ incide en $w$, por lo que $v$ es el único vértice
			que incide en $w$.\\
			Por esto en $G-\{e\}$, ningún vértice incide en $w$. Por lo tanto
			no hay ningún camino que llega a $w$.
			Por lo tanto, $G-\{e\}$ no puede tener raíz.\\

			$d \implies e$\\
			Sea $G$ una digráfica que tiene un raíz $r$ y cualquier arista que quites
			hace que $G$ ya no tenga raíz.\\
			Consideremos la versión no dirigida de $G$, digamos $G'$. Entonces, se
			mantiene que existen caminos $c_u = (u_0 = r, ..., u_n = u)$
			entre $r$ y cualquier otro vértice $u$ de $G'$.\\
			Entonces, entre cualquier $v, w \in V(G')$ se tiene que
			$c = (v = v_n, v_{n-1}, ..., v_1=r=w_1, w_2, ..., w_m)$ es un camino entre
			$v$ y $w$ en $G'$, pues no es dirigida.\\
			Por lo tanto, la verisón no dirigida $G'$ de $G$ es conexa.\\
			Luego, veamos que $\delta_{in}(r) = 0$.
			Supongamos que no, entonces $\delta_{in}(r) \geq 0$. Entonces sea
			$v \in N^+(r)$, por lo que $(v, r)\in E$.\\
			Consideremos a $G - {(v, r)}$, que no debe de tener raíz, por lo que debe
			de existir algún vértice $w$ para el que todo camino en $G$ contenía a $(v, r)$.
			Sea $c = (r = u_1, u_2, ..., u_k = v, u_{k+1} = r, ..., u_n = w)$ uno de
			esos caminos, y que sea de longitud mínima.\\
			Entonces, $c' = (u_{k+1} = r, u_{k+1} = r, ..., u_n = w)$ es un camino entre
			$r$ y $w$ de longitud menos a $c$. Esta contradicción surge de asumir que
			todos los caminos entre $r$ y $v$ pasaban por $(r, v)$. Entonces debe existir
			algún camino que no pasaba por $(r, v)$ en $G$. Entonces, en $G - {(v, r)}$
			sí hay un camino entre $r$ y $w$. Entonces $r$ sigue siendo raíz, pero esto
			no es posible. Esto surge de suponer que $\delta_{in}(r) \geq 0$.
			Entonces $\delta_{in}(r) = 0$.\\
			Ahora veamos que $v$ $\delta(v)_{in} = 1$ para cualquier vértice diferente
			de $r$.\\
			Supongamos que no, entonces sea $(w, v), (u, v) \in E, w \neq u$.\\
			Como $r$ es raíz, entonces existen $c_1 = (r = u_1, ..., u_n = u, v)$ y
			$c_2 = (r = w_1, ..., w_n=w, v)$, digamos de longitud mínima. Notemos que
			son trayectorias, pues de repetir alguna vértice $a$ se podría
			crear otro camino que no incluya a todos los vértices en la sucesión entre
			la primera y la segunda aparición de $a$ que tendría longitus menor.\\
			Consideremos $G - {(w, v)}$, que no debe de tener raíz. Entonces debe de
			existir algún vértice $x$ tal que todos sus caminos en $G$ pasan por
			$(w, v)$. Sea $c_3 = (r = x_1, ..., x_j = w, v = x_{j+1}, ..., x_n = x)$
			unos de esos caminos, y de longitud mínima.\\
			Pero tenemos que $c_4 = (r = u_1, ..., u_n = u, v = x_{j+1}, x_{j+2}, ..., x_n = x)$
			es un camino entre $r$ y $x$, y como ambos son trayectorias y ambos contienen
			a $v$ y en esa aparición de $v$ no está $w$ pegado a $v$, entonces no
			contiene a $(w, v)$, lo cual no es posible. Esta contradicción surge de
			suponer que todos los caminos pasan por $(w, v)$. Entonces hay un camino
			entre $r$ y $x$ en  $G - {(w, v)}$, por lo que $r$ sigue siendo raíz en
			$G - {(w, v)}$, lo cuál no es posible.\\
			Esto surge de suponer que $v$ $\delta(v)_{in} > 1$.\\
			Por lo tanto, $v$ $\delta(v)_{in} = 1$ para cualquier $v$ diferente de la
			raíz.\\

			$e \implies a$\\
			Sea $G$ digráfica tal que la versión no dirigida de $G$ es conexa y $G$
			tiene un vértice $r$ para el cual $\delta_{in}(r) = 0$ mientras que para
			cualquier otro vértice $v$ $\delta(v)_{in} = 1$.\\
			Primero, veamos que $r$ es raíz.\\
			Como la versión no dirigida de $G$ es conexa, entones ahí existe un camino
			entre $r$ y cualquier otro vértice $v$, digamos $c = (r = v_1, ..., v_n = v)$.
			Veamos que este también es un camino entre $r$ y $v$ en $G$.\\
			Como $\delta_{in}(r) = 0$, entonces $r$ no puede tener vértices que le incidan,
			por lo tanto $(r, v_2) \in V(G)$. Y como $\delta(v)_{in} = 1$, y $v_2$ ya
			tiene una arista que le incide, debe de pasa que $(v_2, v_3) \in V(G)$.\\
			Y así sucesivamente. Entonces $(v_j, v_{j+1}) \in V(G), 0 \leq j < n$.\\
			Entonces, efectivamente $c$ es un camino entre $r$ y $v$ en $G$.\\
			Por lo tanto, existe un camino entre $r$ y cualquier vértice $v$.\\
			Entonces $r$ es raíz.\\
			Luego, veamos que la versión no dirigida de $G$ es un árbol.\\
			Como la versión no dirigida de $G$, $G'$ es conexa, sólo falta mostrar que
			$G'$ no tiene ciclos.\\
			Supongamos que tiene algún ciclo $c = (v_1, ..., v_n, v_1)$ en $G'$.\\
			Entonces, es $G$ este ciclo puede ser de varias formas. Digamos $c$ también
			es un ciclo válido en $G$, es decir que $(v_j, v_{j+1}) \in E(G)$.\\
			No puede incluir a $r$, pues de ser así un vértice incidiría en $r$, lo
			cual no es posible. Entonces, para tener un camino entre $r$ y alguno de
			los vértices de $c$, alguno de los vértices del camino debe de tener una
			adyacencia con un vértice que no sea del ciclo, además del vértice al que
			es adyacente dentro del ciclo. Entonces tendría al menos dos vértices que
			le son adyacentes, lo cual no es posible. Entonces el $c$ no puede estar
			en $G'$.\\
			Entonces, si los no se cumple que $(v_j, v_{j+1}) \in E(G), 0 \leq j < n$,
			entonces existe algún $v_j \in V(G)$ que no cumples que
			$(v_j, v_{j+1})\in E(G)$. Y como $v_j$ y $v_{j+1}$ son adyacente en $G'$,
			tiene que pasar que $(v_{j+1}, v_j)\in E(G)$.
			Pero además $(v_{j-1}, v_j)\in E(G)$, por lo que $\delta(v)_{in} \geq 2$.
			Lo cuál no es posible, por lo que $c$ no puede existir en $G'$.
			Entonces, $G´$ es acíclica, por lo que es un árbol.
			Entonces, $G$ tiene raíz y su versión acíclica es un árbol, por lo que es
			$G$ es un árbol dirigido.

		}
	\end{enumerate}
\end{document}
