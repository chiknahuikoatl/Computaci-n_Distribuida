\documentclass[12pt,a4paper]{report}

\usepackage[left=2cm, right=2cm, top=4cm, bottom=2cm]{geometry}
\usepackage{enumitem}
\usepackage{fontspec}
\usepackage{tikz}
\usepackage{amsmath}
\usepackage{algpseudocode}

%Eventos
\algblockdefx[Initially]{Initially}{EndInitially}{\textbf{initially do}}{\textbf{end initially}}
\algblockdefx[Upon]{Upon}{EndUpon}[1]{\textbf{upon #1}}{\textbf{end upon}}

\begin{document}
	%Portada
\begin{titlepage}
	\centering
	{\scshape\LARGE Universidad Nacional Autónoma de México \par}
	\vspace{1cm}
	{\scshape\Large Computación Distribuida\par}
	\vspace{1.5cm}
	{\huge\bfseries Tarea 8\par}
	\vspace{.5cm}
	{\Large\itshape Edgar Quiroz Castañeda \par}
	\vspace{.5cm}
	{\Large\itshape Jerónimo Almeida Rodríguez \par}
	\vfill
	 \includegraphics[width=0.5\textwidth]{escudo_f-ciencias.png}
	\vfill

	{\large Jueves 8 de noviembre del 2018 \par}
\end{titlepage}

\pagebreak
\setlength{\voffset}{-0.75in}
\setlength{\headsep}{5pt}

%Ejercicios
\begin{itemize}
    \item[1]{Considere el algoritmo 3PC visto en clase. Indique el mínimo y el
        máximo número de rondas posibles en una ejecución del algoritmo (cada
        ronda consiste en una fase de env ́ıo de mensajes, ya sea del
        coordinador o de los demas).
    }
    \item[2]{Describa 2 ejecuciones diferentes en las que el algoritmo de 3PC
        falla.
    }
    \item[3]{Explique 5 razones por las cuales es deseable evitar programar con
        timeouts.
    }
    \item[4]{Explique de manera intuitiva la demostracion de correctez del
        algoritmo de Consenso basado en S mostrado en la figura dada.
    }
    \item[5]{Presente una ejecucion con $t=\tfrac{n}{2}$ en la que el algoritmo
        de la figura dada falla.
    }
    \item[6]{Lea los siguientes 5 sueños de Einstein del libro.}
\end{itemize}
\end{document}
