\documentclass[a4paper,12pt]{article}
\usepackage[a4paper, portrait, margin=0.5in]{geometry}
\usepackage[utf8]{inputenc}
\usepackage[spanish]{babel}
\selectlanguage{spanish}
\usepackage{amsmath}
\usepackage{amsfonts}
\usepackage{lmodern}

\begin{document}
\begin{center}
	\section*{Computación Distribuida}
	\section*{\textit{Almeida Rodríguez Jerónimo, Quiroz Castañeda Edgar}}
\end{center}
\section{Moshe Y. Vardi - A Logical Revolution}


\section{Tom Froese - The problem of meaning in AI}
El tema de esta plática fue "qué tan cerca (o lejos) estamos de lograr un sistema que sea verdaderamente inteligente". El dr. Froese inició su discurso hablando de los vehículo autónomos de Uber, uno de ellos recientemente atropelló a una mujer por no poderla identificar, y de la descontinuación del proyecto de ASIMO de honda. El argumento fue que no entendían lo que tenían alrededor.
El dr. Froese planteaba que estos problemas surgen a raíz de que se está intentando resolver el problema de la inteligencia cómo si fuera un problema determinista, pero él cree que puede que sea un problema no determinista. Cómo decía David Hume, saber mucho de un tema no te da forzosamente una acción al respecto.\\
Proponía la existencia de una normatividad

\section{Sergio Rajsbaum - A model of dinamic epistemic logic for fault tolerant distributed computing}

\end{document}
