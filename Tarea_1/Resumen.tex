\documentclass[a4paper,12pt]{article}
\usepackage[a4paper, portrait, margin=0.5in]{geometry}
\usepackage[utf8]{inputenc}
\usepackage[spanish]{babel}
\selectlanguage{spanish}
\usepackage{amsmath}
\usepackage{amsfonts}
\usepackage{lmodern}

\begin{document}
\begin{center}
	\section*{Computación Distribuida}
	\section*{\textit{Almeida Rodríguez Jerónimo, Quiroz Castañeda Edgar}}
\end{center}
\section{Moshe Y. Vardi - A Logical Revolution}


\section{Tom Froese - The problem of meaning in AI}
El tema de esta plática fue "qué tan cerca (o lejos) estamos de lograr un sistema que sea verdaderamente inteligente". El dr. Froese inició su discurso hablando de uno de los vehículos autónomos de Uber, el cual recientemente atropelló a una mujer por no poderla identificar, de la descontinuación del proyecto del robot autónomo de honda, ASIMO, y de un robot que era director de orquesta en Japón. El argumento era que si bien podían hacer relativamente bien su trabajo en el contexto asignado, no entendían su entorno y fuera de él, era obvio que no eran máquinas inteligentes.
El dr. Froese planteaba que estos problemas surgen a raíz de que se está intentando resolver el problema de la inteligencia cómo si fuera un problema determinista, pero él cree que puede que sea un problema no determinista. Parafraseando a David Hume, saber mucho de un tema no te da una acción en respuesta.\\
Froese proponía la existencia de una normatividad

\section{Sergio Rajsbaum - A simplicial complex model of dynamic epistemic logic for fault-tolerant distributed computing}

\end{document}
