\documentclass[a4paper,12pt]{article}
%\usepackage[a4paper, portrait, margin=0.5in]{geometry}
\usepackage[utf8]{inputenc}
\usepackage[spanish]{babel}
\selectlanguage{spanish}
\usepackage{amsmath}
\usepackage{amsfonts}
\usepackage{lmodern}

\begin{document}
\begin{center}
	\section*{Computación Distribuida}
	\section*{\textit{Almeida Rodríguez Jerónimo}}
	\section*{\textit{Quiroz Castañeda Edgar}}
\end{center}


\section{Tom Froese - The problem of meaning in AI}
Uno de los problemas más grandes en la inteligencia artificial es como lograr la representación de conocimiento de forma artificial. Esto sigue siendo una gran barrera, pues incluso hoy en día se desconoce como se almacena el conocimiento de manera natural en los seres inteligentes.\\
En los años 70, se planteó cómo manera de emular la inteligencia el cálculo simbólico, pero con el tiempo se vio que este, al ser escalado, produce resultados no tan deseables.\\
Inclusive, hace poco se suspendieron los proyectos de vehículos autonomos de Uber y la línea de robots ASIMO de Honda. Esto fue debido a la incapacidad de los sistemas de interpretar su entorno en casos especialmente caoticos. Aunque podían hacer relativamente bien su trabajo en el contexto asignado, no entendían su entorno lo suficiente como para actuar de manera inteligente fuera del contexot asignado. El Dr. Froese planteaba que estos problemas surgen a raíz de que se está intentando resolver el problema de la inteligencia cómo si fuera un problema determinista, pero él cree que puede que sea un problema no determinista.
Desde un punto de vista más filośofico, se planteó que puede ser que el enfoque actual a los sistemas inteligentes esté incompleto, ya que el hecho de tener una gran cantidad de datos sobre cierto tema no nos dice como actuar al respecto cuando se presente algún problema relacionado con el mismo.\\
Según la línea de investigación actual de la ciencia cognitiva, puede ser que exista una normatividad, independiente del entorno. Esto sería un comportamiento intrínseco para garantizar la supervivencia del organismo, incluive en organismos tan simples como células. \\
Por otra parte, además de la consistente falla de los sistemas de inteligencia artificial y nuestra ignorancia sobre el funcionamiento del cerebro, no se tiene mucha evidencia acerca de la existencia de esta normatividad.\\
Digamos que la normatividad no existe. Entonces todos las acciones están regidas por principios mecánicos deterministas. Entonces los fenómenos atribuidos a la normatividad no son mas que maneras de explicar aquellos que aún no podemos entender, pero que tienen una explicación totalmente externa al ser inteligente.
Podría ser también que la normatividad existe como algo que influencía el comportamiento, pero que no es lo fundamental que rige el comportamiento. Esto significa, que los organismos serían capaces de pensar acerca de lo que desean, pero que sus acciones, en mayor parte, son mecanicas.\\
También podría ser que la normatividad es lo suficientemente fuerte como para ser determinante en todas las acciones que se toman.\\
En los dos últimos casos, se tendría que hay cierta incertidumbre en las desiciones de seres inteligentes, que no se podría modelar por medio de adquisisción de datos.\\
Y esto aplicado a la computación ha generado varios tipos de modelos.
DiPablo y Pfeifer proponen un sistema dinámico dónde la intención se pódría dar cómo una funcion delimitante que determinaría que acción sería correcta y cual no.\\
Otra vez DiPablo, pero ahora con Jonas propenen que puede ser que la normatividad surja del estado de estar produciendo algo, pues en el caso de los seres vivos si no producen algo, se mueren. En este caso surge el problema de generar sistemas inteligentes que generen algo.

\end{document}
